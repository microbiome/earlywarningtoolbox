\nonstopmode{}
\documentclass[a4paper]{book}
\usepackage[times,inconsolata,hyper]{Rd}
\usepackage{makeidx}
\usepackage[utf8,latin1]{inputenc}
% \usepackage{graphicx} % @USE GRAPHICX@
\makeindex{}
\begin{document}
\chapter*{}
\begin{center}
{\textbf{\huge Package `earlywarnings'}}
\par\bigskip{\large \today}
\end{center}
\begin{description}
\raggedright{}
\item[Type]\AsIs{Package}
\item[Title]\AsIs{Early Warning Signals Toolbox for Detecting Critical Transitions
in Timeseries}
\item[Version]\AsIs{1.0.3}
\item[Date]\AsIs{2013-02-13}
\item[Author]\AsIs{Vasilis Dakos }\email{vasilis.dakos@gmail.com}\AsIs{, with contributions from S.R.
Carpenter, T. Cline, L. Lahti}
\item[Maintainer]\AsIs{Vasilis Dakos }\email{vasilis.dakos@gmail.com}\AsIs{}
\item[Description]\AsIs{The Early-Warning-Signals Toolbox provides methods for estimating
statistical changes in timeseries that can be used for identifying nearby
critical transitions. Based on Dakos et al (2012) Methods for Detecting
Early Warnings of Critical Transitions in Time Series Illustrated Using
Simulated Ecological Data. PLoS ONE 7(7):e41010}
\item[Depend]\AsIs{R (>+ 2.14.0), lmtest, nortest, stats, som, Kendall,
KernSmooth, moments, fields, spam, tseries, quadprog, fractal,
akima, ggplot2}
\item[LazyLoad]\AsIs{yes}
\item[URL]\AsIs{}\url{http://www.early-warning-signals.org,}\AsIs{ }\url{http://www.vasilisdakos.net}\AsIs{}
\item[License]\AsIs{GPL (>= 2)}
\item[Collate]\AsIs{'BDSboot.R' 'bdstest\_ews.R' 'ch\_ews.R' 'ddjnonparam\_ews.R'
'earlywarnings-internal.R' 'generic\_ews.R' 'sensitivity\_ews.R'
'surrogates\_ews.R' 'potential\_ews.R'}
\end{description}
\Rdcontents{\R{} topics documented:}
\inputencoding{utf8}
\HeaderA{bdstest\_ews}{Description: BDS test Early Warning Signals}{bdstest.Rul.ews}
\keyword{early-warning}{bdstest\_ews}
%
\begin{Description}\relax
\code{bdstest\_ews} is used to estimate the BDS statistic
to detect nonlinearity in the residuals of a timeseries
after first-difference detrending, fitting an ARMA(p,q)
model, and fitting a GARCH(0,1) model. The function is
making use of \code{bds.test}.
\end{Description}
%
\begin{Usage}
\begin{verbatim}
  bdstest_ews(timeseries, ARMAoptim = TRUE,
    ARMAorder = c(1, 0), GARCHorder = c(0, 1), embdim = 3,
    epsilon = c(0.5, 0.75, 1), boots = 1000,
    logtransform = FALSE, interpolate = FALSE)
\end{verbatim}
\end{Usage}
%
\begin{Arguments}
\begin{ldescription}
\item[\code{timeseries}] a numeric vector of the observed
univariate timeseries values or a numeric matrix where
the first column represents the time index and the second
the observed timeseries values. Use vectors/matrices with
headings.

\item[\code{ARMAoptim}] is the order of the \code{ARMA(p,q)}
model to be fitted on the original timeseries. If TRUE
the best ARMA model based on AIC is applied. If FALSE the
\code{ARMAorder} is used.

\item[\code{ARMAorder}] is the order of the \code{AR(p)} and
\code{MA(q)} process to be fitted on the original
timeseries. Default is \code{p=1} \code{q=0}.

\item[\code{GARCHorder}] fits a GARCH model on the original
timeseries where \code{GARCHorder[1]} is the GARCH part
and \code{GARCHorder[2]} is the ARCH part.

\item[\code{embdim}] is the embedding dimension (2, 3,...
\code{embdim}) up to which the BDS test will be estimated
(must be numeric). Default value is 3.

\item[\code{epsilon}] is a numeric vector that is used to scale
the standard deviation of the timeseries. The BDS test is
computed for each element of epsilon. Default is 0.5,
0.75 and 1.

\item[\code{boots}] is the number of bootstraps performed to
estimate significance p values for the BDS test. Default
is 1000.

\item[\code{logtransform}] logical. If TRUE data are
logtransformed prior to analysis as log(X+1). Default is
FALSE.

\item[\code{interpolate}] logical. If TRUE linear interpolation
is applied to produce a timeseries of equal length as the
original. Default is FALSE (assumes there are no gaps in
the timeseries).
\end{ldescription}
\end{Arguments}
%
\begin{Details}\relax
See also \code{bds.test\{tseries\}} for more details. The
function requires the installation of packages
\code{tseries} and \code{quadprog} that are not available
under Linux and need to be manually installed under
Windows.

Example to run after installing the mentioned packages:

data(foldbif)
bdstest\_ews(foldbif,ARMAoptim=FALSE,ARMAorder=c(1,0),embdim=3,epsilon=0.5,
boots=200,logtransform=FALSE,interpolate=FALSE)
\end{Details}
%
\begin{Value}
\code{bdstest\_ews} returns output on the R console that
summarizes the BDS test statistic for all embedding
dimensions and \code{epsilon} values used, and for
first-differenced data, ARMA(p.q) residuals, and
GARCH(0,1) residuals). Also the significance p values are
returned estimated both by comparing to a standard normal
distribution and by bootstrapping.

In addition, \code{bdstest\_ews} returns a plot with the
original timeseries, the residuals after
first-differencing, and fitting the ARMA(p,q) and
GARCH(0,1) models. Also the autocorrelation
\code{\LinkA{acf}{acf}} and partial autocorrelation
\code{\LinkA{pacf}{pacf}} functions are estimated serving as
guides for the choice of lags of the linear models fitted
to the data.
\end{Value}
%
\begin{Author}\relax
S. R. Carpenter, modified by V. Dakos
\end{Author}
%
\begin{References}\relax
J. B. Cromwell, W. C. Labys and M. Terraza (1994):
Univariate Tests for Time Series Models, Sage, Thousand
Oaks, CA, pages 32-36.

Dakos, V., et al (2012)."Methods for Detecting Early
Warnings of Critical Transitions in Time Series
Illustrated Using Simulated Ecological Data." \emph{PLoS
ONE} 7(7): e41010. doi:10.1371/journal.pone.0041010
\end{References}
%
\begin{SeeAlso}\relax
\code{\LinkA{generic\_ews}{generic.Rul.ews}}; \code{\LinkA{ddjnonparam\_ews}{ddjnonparam.Rul.ews}};
\code{\LinkA{bdstest\_ews}{bdstest.Rul.ews}}; \code{\LinkA{sensitivity\_ews}{sensitivity.Rul.ews}};
\code{\LinkA{surrogates\_ews}{surrogates.Rul.ews}}; \code{\LinkA{ch\_ews}{ch.Rul.ews}};
\code{\LinkA{movpotential\_ews}{movpotential.Rul.ews}};
\code{\LinkA{livpotential\_ews}{livpotential.Rul.ews}}
\end{SeeAlso}
\inputencoding{utf8}
\HeaderA{ch\_ews}{Description: Conditional Heteroskedasticity}{ch.Rul.ews}
\keyword{early-warning}{ch\_ews}
%
\begin{Description}\relax
\code{ch\_ews} is used to estimate changes in conditional
heteroskedasticity within rolling windows along a
timeseries
\end{Description}
%
\begin{Usage}
\begin{verbatim}
  ch_ews(timeseries, winsize = 10, alpha = 0.1,
    optim = TRUE, lags = 4, logtransform = FALSE,
    interpolate = FALSE)
\end{verbatim}
\end{Usage}
%
\begin{Arguments}
\begin{ldescription}
\item[\code{timeseries}] a numeric vector of the observed
timeseries values or a numeric matrix where the first
column represents the time index and the second the
observed timeseries values. Use vectors/matrices with
headings.

\item[\code{winsize}] is length of the rolling window expressed
as percentage of the timeseries length (must be numeric
between 0 and 100). Default is 10\%.

\item[\code{alpha}] is the significance threshold (must be
numeric). Default is 0.1.

\item[\code{optim}] logical. If TRUE an autoregressive model is
fit to the data within the rolling window using AIC
optimization. Otherwise an autoregressive model of
specific order \code{lags} is selected.

\item[\code{lags}] is a parameter that determines the specific
order of an autoregressive model to fit the data. Default
is 4.

\item[\code{logtransform}] logical. If TRUE data are
logtransformed prior to analysis as log(X+1). Default is
FALSE.

\item[\code{interpolate}] logical. If TRUE linear interpolation
is applied to produce a timeseries of equal length as the
original. Default is FALSE (assumes there are no gaps in
the timeseries).
\end{ldescription}
\end{Arguments}
%
\begin{Details}\relax
see ref below
\end{Details}
%
\begin{Value}
\code{ch\_ews} returns a matrix that contains:

\begin{ldescription}
\item[\code{time}] the time index.

\item[\code{r.squared}] the R2 values of the regressed
residuals.

\item[\code{critical.value}] the chi-square critical value based
on the desired \code{alpha} level for 1 degree of freedom
divided by the number of residuals used in the
regression.

\item[\code{test.result}] logical. It indicates whether
conditional heteroskedasticity was significant.

\item[\code{ar.fit.order}] the order of the specified
autoregressive model- only informative if \code{optim}
FALSE was selected.

\end{ldescription}
In addition, \code{ch\_ews} plots the original timeseries
and the R2 where the level of significance is also
indicated.
\end{Value}
%
\begin{Author}\relax
T. Cline, modified by V. Dakos
\end{Author}
%
\begin{References}\relax
Seekell, D. A., et al (2011). "Conditional
heteroscedasticity as a leading indicator of ecological
regime shifts." \emph{American Naturalist} 178(4):
442-451

Dakos, V., et al (2012)."Methods for Detecting Early
Warnings of Critical Transitions in Time Series
Illustrated Using Simulated Ecological Data." \emph{PLoS
ONE} 7(7): e41010. doi:10.1371/journal.pone.0041010
\end{References}
%
\begin{SeeAlso}\relax
\code{\LinkA{generic\_ews}{generic.Rul.ews}}; \code{\LinkA{ddjnonparam\_ews}{ddjnonparam.Rul.ews}};
\code{\LinkA{bdstest\_ews}{bdstest.Rul.ews}}; \code{\LinkA{sensitivity\_ews}{sensitivity.Rul.ews}};
\code{\LinkA{surrogates\_ews}{surrogates.Rul.ews}}; \code{\LinkA{ch\_ews}{ch.Rul.ews}};
\code{movpotential\_ews}; \code{livpotential\_ews}
\end{SeeAlso}
%
\begin{Examples}
\begin{ExampleCode}
data(foldbif)
out=ch_ews(foldbif, winsize=50, alpha=0.05, optim=TRUE, lags)
\end{ExampleCode}
\end{Examples}
\inputencoding{utf8}
\HeaderA{circulation}{Overturning of thermolhaline circulation}{circulation}
\keyword{datasets}{circulation}
%
\begin{Description}\relax
Simulated timeseries of salinity from an ocean circulation model

\end{Description}
%
\begin{Usage}
\begin{verbatim}
data(circulation)
\end{verbatim}
\end{Usage}
%
\begin{Format}
A data frame with 783 observations on the following 2 variables.
\begin{description}

\item[\code{time}] a numeric vector
\item[\code{x}] a numeric vector

\end{description}

\end{Format}
%
\begin{Details}\relax
Simulated timeseries of salinity from an ocean circulation model as used in Dakos et al (2008), see source below. At the end of the timeseries a critical transition occurs that simulates the overturning of the thermohalien circulation.

\end{Details}
%
\begin{Source}\relax
Dakos et al (2008), Slowing down as an early warning signal for abrup climate change, PNAS 105(38), 14308-14312.

\end{Source}
\inputencoding{utf8}
\HeaderA{ddjnonparam\_ews}{Description: Drift Diffusion Jump Nonparametrics Early Warning Signals}{ddjnonparam.Rul.ews}
\keyword{early-warning}{ddjnonparam\_ews}
%
\begin{Description}\relax
\code{ddjnonparam\_ews} is used to compute
nonparametrically conditional variance, drift, diffusion
and jump intensity in a timeseries. It also interpolates
to obtain the evolution of the nonparametric statistics
in time.
\end{Description}
%
\begin{Usage}
\begin{verbatim}
  ddjnonparam_ews(timeseries, bandwidth = 0.6, na = 500,
    logtransform = TRUE, interpolate = FALSE)
\end{verbatim}
\end{Usage}
%
\begin{Arguments}
\begin{ldescription}
\item[\code{timeseries}] a numeric vector of the observed
univariate timeseries values or a numeric matrix where
the first column represents the time index and the second
the observed timeseries values. Use vectors/matrices with
headings.

\item[\code{bandwidth}] is the bandwidht of the kernel regressor
(must be numeric). Default is 0.6.

\item[\code{na}] is the number of points for computing the
kernel (must be numeric). Default is 500.

\item[\code{logtransform}] logical. If TRUE data are
logtransformed prior to analysis as log(X+1). Default is
FALSE.

\item[\code{interpolate}] logical. If TRUE linear interpolation
is applied to produce a timeseries of equal length as the
original. Default is FALSE (assumes there are no gaps in
the timeseries).
\end{ldescription}
\end{Arguments}
%
\begin{Details}\relax
The approach is based on estimating terms of a
drift-diffusion-jump model as a surrogate for the unknown
true data generating process: [1] \eqn{dx = f(x,\theta)dt
  + g(x,\theta)dW + dJ}{} Here x is the state variable, f()
and g() are nonlinear functions, dW is a Wiener process
and dJ is a jump process. Jumps are large, one-step,
positive or negative shocks that are uncorrelated in
time.
\end{Details}
%
\begin{Value}
\code{ddjnonparam\_ews} returns an object with elements:

\begin{ldescription}
\item[\code{avec}] is the mesh for which values of the
nonparametric statistics are estimated.

\item[\code{S2.vec}] is the conditional variance of the
timeseries \code{x} over \code{avec}.

\item[\code{TotVar.dx.vec}] is the total variance of \code{dx}
over \code{avec}.

\item[\code{Diff2.vec}] is the diffusion estimated as
\code{total variance - jumping intensity} vs
\code{avec}.

\item[\code{LamdaZ.vec}] is the jump intensity over
\code{avec}.

\item[\code{Tvec1}] is the timeindex.

\item[\code{S2.t}] is the conditional variance of the timeseries
\code{x} data over \code{Tvec1}.

\item[\code{TotVar.t}] is the total variance of \code{dx} over
\code{Tvec1}.

\item[\code{Diff2.t}] is the diffusion over \code{Tvec1}.

\item[\code{Lamda.t}] is the jump intensity over \code{Tvec1}.

\end{ldescription}
In addition, \code{ddjnonparam\_ews} returns a first plot
with the original timeseries and the residuals after
first-differencing. A second plot shows the nonparametric
conditional variance, total variance, diffusion and jump
intensity over the data, and a third plot the same
nonparametric statistics over time.
\end{Value}
%
\begin{Author}\relax
S. R. Carpenter, modified by V. Dakos
\end{Author}
%
\begin{References}\relax
Carpenter, S. R. and W. A. Brock (2011). "Early warnings
of unknown nonlinear shifts: a nonparametric approach."
\emph{Ecology} 92(12): 2196-2201

Dakos, V., et al (2012)."Methods for Detecting Early
Warnings of Critical Transitions in Time Series
Illustrated Using Simulated Ecological Data." \emph{PLoS
ONE} 7(7): e41010. doi:10.1371/journal.pone.0041010
\end{References}
%
\begin{SeeAlso}\relax
\code{\LinkA{generic\_ews}{generic.Rul.ews}}; \code{\LinkA{ddjnonparam\_ews}{ddjnonparam.Rul.ews}};
\code{\LinkA{bdstest\_ews}{bdstest.Rul.ews}};
\code{\LinkA{sensitivity\_ews}{sensitivity.Rul.ews}};\code{\LinkA{surrogates\_ews}{surrogates.Rul.ews}};
\code{\LinkA{ch\_ews}{ch.Rul.ews}}; \code{\LinkA{movpotential\_ews}{movpotential.Rul.ews}};
\code{\LinkA{livpotential\_ews}{livpotential.Rul.ews}}
\end{SeeAlso}
%
\begin{Examples}
\begin{ExampleCode}
data(foldbif)
output<-ddjnonparam_ews(foldbif,bandwidth=0.6,na=500,
logtransform=TRUE,interpolate=FALSE)
\end{ExampleCode}
\end{Examples}
\inputencoding{utf8}
\HeaderA{foldbif}{Simulated fold bifurcation time series}{foldbif}
\keyword{datasets}{foldbif}
%
\begin{Description}\relax
Simulated time series of a transition to an alternative state.

\end{Description}
%
\begin{Usage}
\begin{verbatim}
data(foldbif)
\end{verbatim}
\end{Usage}
%
\begin{Format}
A data frame with 970 observations on the following variable.
\begin{description}

\item[\code{x}] a numeric vector

\end{description}

\end{Format}
%
\begin{Details}\relax
Simulated time series of a transition to an alternative state derived from a simple overharvesting model as used in Dakos et al (2012), see source below.

\end{Details}
%
\begin{Source}\relax
Dakos et al (2012) Methods for Detecting Early Warnings of Critical Transitions in Time Series Illustrated Using Simulated Ecological Data. PLoS ONE 7(7):e41010

\end{Source}
\inputencoding{utf8}
\HeaderA{generic\_ews}{Description: Generic Early Warning Signals}{generic.Rul.ews}
\keyword{early-warning}{generic\_ews}
%
\begin{Description}\relax
\code{generic\_ews} is used to estimate statistical
moments within rolling windows along a timeserie
\end{Description}
%
\begin{Usage}
\begin{verbatim}
  generic_ews(timeseries, winsize = 50,
    detrending = c("no", "gaussian", "linear", "first-diff"),
    bandwidth = NULL, logtransform = FALSE,
    interpolate = FALSE, AR_n = FALSE,
    powerspectrum = FALSE)
\end{verbatim}
\end{Usage}
%
\begin{Arguments}
\begin{ldescription}
\item[\code{timeseries}] a numeric vector of the observed
univariate timeseries values or a numeric matrix where
the first column represents the time index and the second
the observed timeseries values. Use vectors/matrices with
headings. If the powerspectrum is to be plotted as well,
the timeseries lenght should be even number.

\item[\code{winsize}] is the size of the rolling window
expressed as percentage of the timeseries length (must be
numeric between 0 and 100). Default is 50\%.

\item[\code{bandwidth}] is the bandwidth used for the Gaussian
kernel when gaussian filtering is applied. It is
expressed as percentage of the timeseries length (must be
numeric between 0 and 100). Alternatively it can be given
by the bandwidth selector \code{\LinkA{bw.nrd0}{bw.nrd0}}
(Default).

\item[\code{detrending}] the timeseries can be
detrended/filtered prior to analysis. There are four
options: \code{gaussian} filtering, \code{linear}
detrending and \code{first-differencing}. Default is
\code{no} detrending.

\item[\code{logtransform}] logical. If TRUE data are
logtransformed prior to analysis as log(X+1). Default is
FALSE.

\item[\code{interpolate}] logical. If TRUE linear interpolation
is applied to produce a timeseries of equal length as the
original. Default is FALSE (assumes there are no gaps in
the timeseries).

\item[\code{AR\_n}] logical. If TRUE the best fitted AR(n) model
is fitted to the data. Default is FALSE.

\item[\code{powerspectrum}] logical. If TRUE the power spectrum
within each rolling window is plotted. Default is FALSE.
\end{ldescription}
\end{Arguments}
%
\begin{Details}\relax
see ref below
\end{Details}
%
\begin{Value}
\code{generic\_ews} returns a matrix that contains:

\begin{ldescription}
\item[\code{tim}] the time index.

\item[\code{ar1}] the \code{autoregressive coefficient ar(1)} of
a first order AR model fitted on the data within the
rolling window.

\item[\code{sd}] the \code{standard deviation} of the data
estimated within each rolling window.

\item[\code{sk}] the \code{skewness} of the data estimated
within each rolling window.

\item[\code{kurt}] the \code{kurtosis} of the data estimated
within each rolling window.

\item[\code{cv}] the \code{coefficient of variation} of the data
estimated within each rolling window.

\item[\code{returnrate}] the return rate of the data estimated
as \code{1-ar(1)} cofficient within each rolling window.

\item[\code{densratio}] the \code{density ratio} of the power
spectrum of the data estimated as the ratio of low
frequencies over high frequencies within each rolling
window.

\item[\code{acf1}] the \code{autocorrelation at first lag} of
the data estimated within each rolling window.

\end{ldescription}
In addition, \code{generic\_ews} returns three plots. The
first plot contains the original data, the
detrending/filtering applied and the residuals (if
selected), and all the moment statistics. For each
statistic trends are estimated by the nonparametric
Kendall tau correlation.  The second plot, if asked,
quantifies resilience indicators fitting AR(n) selected
by the Akaike Information Criterion. The third plot, if
asked, is the power spectrum estimated by
\code{\LinkA{spec.ar}{spec.ar}} for all frequencies within each
rolling window.
\end{Value}
%
\begin{Author}\relax
Vasilis Dakos \email{vasilis.dakos@gmail.com}
\end{Author}
%
\begin{References}\relax
Ives, A. R. (1995). "Measuring resilience in stochastic
systems." \emph{Ecological Monographs} 65: 217-233

Dakos, V., et al (2008). "Slowing down as an early
warning signal for abrupt climate change."
\emph{Proceedings of the National Academy of Sciences}
105(38): 14308-14312

Dakos, V., et al (2012)."Methods for Detecting Early
Warnings of Critical Transitions in Time Series
Illustrated Using Simulated Ecological Data." \emph{PLoS
ONE} 7(7): e41010. doi:10.1371/journal.pone.0041010
\end{References}
%
\begin{SeeAlso}\relax
\code{\LinkA{generic\_ews}{generic.Rul.ews}}; \code{\LinkA{ddjnonparam\_ews}{ddjnonparam.Rul.ews}};
\code{\LinkA{bdstest\_ews}{bdstest.Rul.ews}}; \code{\LinkA{sensitivity\_ews}{sensitivity.Rul.ews}};
\code{\LinkA{surrogates\_ews}{surrogates.Rul.ews}}; \code{\LinkA{ch\_ews}{ch.Rul.ews}};
\code{\LinkA{movpotential\_ews}{movpotential.Rul.ews}};
\code{\LinkA{livpotential\_ews}{livpotential.Rul.ews}}
\end{SeeAlso}
%
\begin{Examples}
\begin{ExampleCode}
data(foldbif)
 out=generic_ews(foldbif,winsize=50,detrending="gaussian",
 bandwidth=5,logtransform=FALSE,interpolate=FALSE)
\end{ExampleCode}
\end{Examples}
\inputencoding{utf8}
\HeaderA{livpotential\_ews}{Description: Potential Analysis}{livpotential.Rul.ews}
\keyword{early-warning}{livpotential\_ews}
%
\begin{Description}\relax
\code{livpotential\_ews} performs one-dimensional
potential estimation derived from a uni-variate
timeseries
\end{Description}
%
\begin{Usage}
\begin{verbatim}
  livpotential_ews(x, std = 1, bw = -1, xi = NULL,
    weights = c(), grid.size = 200)
\end{verbatim}
\end{Usage}
%
\begin{Arguments}
\begin{ldescription}
\item[\code{x}] data vector

\item[\code{std}] the standard deviation of the noise (defaults
to 1, so then you use scaled potentials

\item[\code{bw}] bandwidth for kernel estimation

\item[\code{xi}] x values at which the potential is estimated

\item[\code{weights}] optional weights in ksdensity (used by
movpotentials).

\item[\code{grid.size}] grid size
\end{ldescription}
\end{Arguments}
%
\begin{Details}\relax
see ref below
\end{Details}
%
\begin{Value}
\code{livpotential} returns a list with the following
elements:

\begin{ldescription}
\item[\code{xi}] the grid of points on which the potential is
estimated

\item[\code{pot}] the actual value of the potential

\item[\code{minima}] the grid points at which the potential has
minimum values

\item[\code{maxima}] the grid points at which the potential has
maximum values

\item[\code{bw}] bandwidth of kernel used
\end{ldescription}
\end{Value}
%
\begin{Author}\relax
Based on Matlab code from Egbert van Nes modified by Leo
Lahti. Implemented in early warnings package by V. Dakos.
\end{Author}
%
\begin{References}\relax
Livina, VN, F Kwasniok, and TM Lenton, 2010. Potential
analysis reveals changing number of climate states during
the last 60 kyr . \emph{Climate of the Past}, 6, 77-82.

Dakos, V., et al (2012)."Methods for Detecting Early
Warnings of Critical Transitions in Time Series
Illustrated Using Simulated Ecological Data." \emph{PLoS
ONE} 7(7): e41010. doi:10.1371/journal.pone.0041010
\end{References}
%
\begin{SeeAlso}\relax
\code{\LinkA{generic\_ews}{generic.Rul.ews}}; \code{\LinkA{ddjnonparam\_ews}{ddjnonparam.Rul.ews}};
\code{\LinkA{bdstest\_ews}{bdstest.Rul.ews}};
\code{\LinkA{sensitivity\_ews}{sensitivity.Rul.ews}};\code{\LinkA{surrogates\_ews}{surrogates.Rul.ews}};
\code{\LinkA{ch\_ews}{ch.Rul.ews}};\code{\LinkA{movpotential\_ews}{movpotential.Rul.ews}}
\end{SeeAlso}
%
\begin{Examples}
\begin{ExampleCode}
data(foldbif)
res <- livpotential_ews(foldbif)
plot(res$xi, res$pot)
\end{ExampleCode}
\end{Examples}
\inputencoding{utf8}
\HeaderA{movpotential\_ews}{Description: Moving Average Potential}{movpotential.Rul.ews}
\keyword{early-warning}{movpotential\_ews}
%
\begin{Description}\relax
\code{movpotential\_ews} reconstructs a potential derived
from data along a gradient of a given parameter the
\code{movpotential\_ews} calculates the potential for
values that correspond to a particular parameter. see ref
below
\end{Description}
%
\begin{Usage}
\begin{verbatim}
  movpotential_ews(X, param, sdwindow = NULL, bw = -1,
    minparam = NULL, maxparam = NULL, npoints = 50,
    thres = 0.002, std = 1, grid.size = 200, cutoff = 0.5)
\end{verbatim}
\end{Usage}
%
\begin{Arguments}
\begin{ldescription}
\item[\code{X}] a vector of the X observations of the state
variable of interest

\item[\code{param}] parameter values that correspond to the X
observations

\item[\code{sdwindow}] window for smoothing kernels (over the
\code{param} axis)

\item[\code{bw}] bandwidth used for smoothing kernels

\item[\code{minparam}] minimum value of parameter on which to
estimate potential

\item[\code{maxparam}] maximum value of parameter on which to
estimate potential

\item[\code{npoints}] number of potentials

\item[\code{thres}] threshold for local minima to be discarded

\item[\code{std}] std

\item[\code{grid.size}] number of evaluation points

\item[\code{cutoff}] the cuttof value to estimate minima and
maxima in the potential

Returns:
\end{ldescription}
\end{Arguments}
%
\begin{Value}
A list with the following elements:

\begin{ldescription}
\item[\code{pars}] values of the covariate parameter as matrix

\item[\code{xis}] values of the x as matrix

\item[\code{pots}] smoothed potentials

\item[\code{mins}] minima in the densities (-potentials;
neglecting local optima)

\item[\code{maxs}] maxima in densities (-potentials; neglecting
local optima)

\item[\code{plot}] an object that displays the potential
estimated in 2D
\end{ldescription}
\end{Value}
%
\begin{Author}\relax
Based on Matlab code from Egbert van Nes modified by Leo
Lahti. Implemented in early warnings package by V. Dakos.
\end{Author}
%
\begin{References}\relax
Hirota, M., Holmgren, M., van Nes, E.H. \& Scheffer, M.
(2011). Global resilience of tropical forest and savanna
to critical transitions. \emph{Science}, 334, 232-235.
\end{References}
%
\begin{SeeAlso}\relax
\code{\LinkA{generic\_ews}{generic.Rul.ews}}; \code{\LinkA{ddjnonparam\_ews}{ddjnonparam.Rul.ews}};
\code{\LinkA{bdstest\_ews}{bdstest.Rul.ews}};
\code{\LinkA{sensitivity\_ews}{sensitivity.Rul.ews}};\code{\LinkA{surrogates\_ews}{surrogates.Rul.ews}};
\code{\LinkA{ch\_ews}{ch.Rul.ews}}; \code{livpotential\_ews}
\end{SeeAlso}
%
\begin{Examples}
\begin{ExampleCode}
X = c(rnorm(1000, mean = 0), rnorm(1000, mean = -2), rnorm(1000, mean = 2))
 param = seq(0,5,length=3000)
 res <- movpotential_ews(X, param, npoints = 100, thres = 0.003)
\end{ExampleCode}
\end{Examples}
\inputencoding{utf8}
\HeaderA{sensitivity\_ews}{Description: Sensitivity Early Warning Signals}{sensitivity.Rul.ews}
\keyword{early-warning}{sensitivity\_ews}
%
\begin{Description}\relax
\code{sensitivity\_ews} is used to estimate trends in
statistical moments for different sizes of rolling
windows along a timeseries. The trends are estimated by
the nonparametric Kendall tau correlation coefficient.
\end{Description}
%
\begin{Usage}
\begin{verbatim}
  sensitivity_ews(timeseries,
    indicator = c("ar1", "sd", "acf1", "sk", "kurt", "cv", "returnrate", "densratio"),
    winsizerange = c(25, 75), incrwinsize = 25,
    detrending = c("no", "gaussian", "linear", "first-diff"),
    bandwidthrange = c(5, 100), incrbandwidth = 20,
    logtransform = FALSE, interpolate = FALSE)
\end{verbatim}
\end{Usage}
%
\begin{Arguments}
\begin{ldescription}
\item[\code{timeseries}] a numeric vector of the observed
univariate timeseries values or a numeric matrix where
the first column represents the time index and the second
the observed timeseries values. Use vectors/matrices with
headings.

\item[\code{indicator}] is the statistic (leading indicator)
selected for which the sensitivity analysis is perfomed.
Currently, the indicators supported are: \code{ar1}
autoregressive coefficient of a first order AR model,
\code{sd} standard deviation, \code{acf1} autocorrelation
at first lag, \code{sk} skewness, \code{kurt} kurtosis,
\code{cv} coeffcient of variation, \code{returnrate}, and
\code{densratio} density ratio of the power spectrum at
low frequencies over high frequencies.

\item[\code{winsizerange}] is the range of the rolling window
sizes expressed as percentage of the timeseries length
(must be numeric between 0 and 100). Default is 25\% -
75\%.

\item[\code{incrwinsize}] increments the rolling window size
(must be numeric between 0 and 100). Default is 25.

\item[\code{detrending}] the timeseries can be
detrended/filtered. There are three options:
\code{gaussian} filtering, \code{linear} detrending and
\code{first-differencing}. Default is \code{no}
detrending.

\item[\code{bandwidthrange}] is the range of the bandwidth used
for the Gaussian kernel when gaussian filtering is
selected. It is expressed as percentage of the timeseries
length (must be numeric between 0 and 100). Default is
5\% - 100\%.

\item[\code{incrbandwidth}] is the size to increment the
bandwidth used for the Gaussian kernel when gaussian
filtering is applied. It is expressed as percentage of
the timeseries length (must be numeric between 0 and
100). Default is 20.

\item[\code{logtransform}] logical. If TRUE data are
logtransformed prior to analysis as log(X+1). Default is
FALSE.

\item[\code{interpolate}] logical. If TRUE linear interpolation
is applied to produce a timeseries of equal length as the
original. Default is FALSE (assumes there are no gaps in
the timeseries).
\end{ldescription}
\end{Arguments}
%
\begin{Details}\relax
see ref below
\end{Details}
%
\begin{Value}
\code{sensitivity\_ews} returns a matrix that contains the
Kendall tau rank correlation estimates for the rolling
window sizes (rows) and bandwidths (columns), if
\code{gaussian filtering} is selected.

In addition, \code{sensitivity\_ews} returns a plot with
the Kendall tau estimates and their p-values for the
range of rolling window sizes used, together with a
histogram of the distributions of the statistic and its
significance. When \code{gaussian filtering} is chosen, a
contour plot is produced for the Kendall tau estimates
and their p-values for the range of both rolling window
sizes and bandwidth used. A reverse triangle indicates
the combination of the two parameters for which the
Kendall tau was the highest
\end{Value}
%
\begin{Author}\relax
Vasilis Dakos \email{vasilis.dakos@gmail.com}
\end{Author}
%
\begin{References}\relax
Dakos, V., et al (2008). "Slowing down as an early
warning signal for abrupt climate change."
\emph{Proceedings of the National Academy of Sciences}
105(38): 14308-14312

Dakos, V., et al (2012)."Methods for Detecting Early
Warnings of Critical Transitions in Time Series
Illustrated Using Simulated Ecological Data." \emph{PLoS
ONE} 7(7): e41010. doi:10.1371/journal.pone.0041010
\end{References}
%
\begin{SeeAlso}\relax
\code{\LinkA{generic\_ews}{generic.Rul.ews}}; \code{\LinkA{ddjnonparam\_ews}{ddjnonparam.Rul.ews}};
\code{\LinkA{bdstest\_ews}{bdstest.Rul.ews}}; \code{\LinkA{sensitivity\_ews}{sensitivity.Rul.ews}};
\code{\LinkA{surrogates\_ews}{surrogates.Rul.ews}}; \code{\LinkA{ch\_ews}{ch.Rul.ews}};
\code{\LinkA{movpotential\_ews}{movpotential.Rul.ews}};
\code{\LinkA{livpotential\_ews}{livpotential.Rul.ews}}
\end{SeeAlso}
%
\begin{Examples}
\begin{ExampleCode}
data(foldbif)
output=sensitivity_ews(foldbif,indicator="sd",detrending="gaussian",
incrwinsize=25,incrbandwidth=20)
\end{ExampleCode}
\end{Examples}
\inputencoding{utf8}
\HeaderA{surrogates\_ews}{Description: Surrogates Early Warning Signals}{surrogates.Rul.ews}
\keyword{early-warning}{surrogates\_ews}
%
\begin{Description}\relax
\code{surrogates\_ews} is used to estimate distributions
of trends in statistical moments from different surrogate
timeseries generated after fitting an ARMA(p,q) model on
the data. The trends are estimated by the nonparametric
Kendall tau correlation coefficient and can be compared
to the trends estimated in the original timeseries to
produce probabilities of false positives.
\end{Description}
%
\begin{Usage}
\begin{verbatim}
  surrogates_ews(timeseries,
    indicator = c("ar1", "sd", "acf1", "sk", "kurt", "cv", "returnrate", "densratio"),
    winsize = 50,
    detrending = c("no", "gaussian", "linear", "first-diff"),
    bandwidth = NULL, boots = 100, logtransform = FALSE,
    interpolate = FALSE)
\end{verbatim}
\end{Usage}
%
\begin{Arguments}
\begin{ldescription}
\item[\code{timeseries}] a numeric vector of the observed
univariate timeseries values or a numeric matrix where
the first column represents the time index and the second
the observed timeseries values. Use vectors/matrices with
headings.

\item[\code{indicator}] is the statistic (leading indicator)
selected for which the surrogate timeseries are produced.
Currently, the indicators supported are: \code{ar1}
autoregressive coefficient of a first order AR model,
\code{sd} standard deviation, \code{acf1} autocorrelation
at first lag, \code{sk} skewness, \code{kurt} kurtosis,
\code{cv} coeffcient of variation, \code{returnrate}, and
\code{densratio} density ratio of the power spectrum at
low frequencies over high frequencies.

\item[\code{winsize}] is the size of the rolling window
expressed as percentage of the timeseries length (must be
numeric between 0 and 100). Default valuise 50\%.

\item[\code{detrending}] the timeseries can be
detrended/filtered prior to analysis. There are three
options: \code{gaussian} filtering, \code{linear}
detrending and \code{first-diff}erencing. Default is
\code{no} detrending.

\item[\code{bandwidth}] is the bandwidth used for the Gaussian
kernel when gaussian filtering is selected. It is
expressed as percentage of the timeseries length (must be
numeric between 0 and 100). Alternatively it can be given
by the bandwidth selector \code{\LinkA{bw.nrd0}{bw.nrd0}}
(Default).

\item[\code{boots}] the number of surrogate data. Default is
100.

\item[\code{logtransform}] logical. If TRUE data are
logtransformed prior to analysis as log(X+1). Default is
FALSE.

\item[\code{interpolate}] logical. If TRUE linear interpolation
is applied to produce a timeseries of equal length as the
original. Default is FALSE (assumes there are no gaps in
the timeseries).
\end{ldescription}
\end{Arguments}
%
\begin{Details}\relax
see ref below
\end{Details}
%
\begin{Value}
\code{surrogates\_ews} returns a matrix that contains:

\begin{ldescription}
\item[\code{Kendall tau estimate original}] the trends of the
original timeseries.

\item[\code{Kendall tau p-value original}] the p-values of the
trends of the original timeseries.

\item[\code{Kendall tau estimate surrogates}] the trends of the
surrogate timeseries.

\item[\code{Kendall tau p-value surrogates}] the associated
p-values of the trends of the surrogate timeseries.

\item[\code{significance p}] the p-value for the original
Kendall tau rank correlation estimate compared to the
surrogates.

\end{ldescription}
In addition, \code{surrogates\_ews} returns a plot with
the distribution of the surrogate Kendall tau estimates
and the Kendall tau estimate of the original series.
Vertical lines indicate the 5\% and 95\% significance
levels.
\end{Value}
%
\begin{Author}\relax
Vasilis Dakos \email{vasilis.dakos@gmail.com}
\end{Author}
%
\begin{References}\relax
Dakos, V., et al (2008). "Slowing down as an early
warning signal for abrupt climate change."
\emph{Proceedings of the National Academy of Sciences}
105(38): 14308-14312

Dakos, V., et al (2012)."Methods for Detecting Early
Warnings of Critical Transitions in Time Series
Illustrated Using Simulated Ecological Data." \emph{PLoS
ONE} 7(7): e41010. doi:10.1371/journal.pone.0041010
\end{References}
%
\begin{SeeAlso}\relax
\code{\LinkA{generic\_ews}{generic.Rul.ews}}; \code{\LinkA{ddjnonparam\_ews}{ddjnonparam.Rul.ews}};
\code{\LinkA{bdstest\_ews}{bdstest.Rul.ews}}; \code{\LinkA{sensitivity\_ews}{sensitivity.Rul.ews}};
\code{\LinkA{surrogates\_ews}{surrogates.Rul.ews}}; \code{\LinkA{ch\_ews}{ch.Rul.ews}};
\code{\LinkA{movpotential\_ews}{movpotential.Rul.ews}};
\code{\LinkA{livpotential\_ews}{livpotential.Rul.ews}}
\end{SeeAlso}
%
\begin{Examples}
\begin{ExampleCode}
data(foldbif);
output=surrogates_ews(foldbif,indicator="sd",winsize=50,detrending="gaussian",
bandwidth=10,boots=200,logtransform=FALSE,interpolate=FALSE)
\end{ExampleCode}
\end{Examples}
\inputencoding{utf8}
\HeaderA{YD2PB\_grayscale}{Younger Dryas to PreBoreal transition}{YD2PB.Rul.grayscale}
\keyword{datasets}{YD2PB\_grayscale}
%
\begin{Description}\relax
Grayscale paleodata of the exit from the Younger Dryas

\end{Description}
%
\begin{Usage}
\begin{verbatim}
data(YD2PB_grayscale)
\end{verbatim}
\end{Usage}
%
\begin{Format}
A data frame with 2111 observations on the following 2 variables.
\begin{description}

\item[\code{time}] a numeric vector
\item[\code{x}] a numeric vector

\end{description}

\end{Format}
%
\begin{Details}\relax
Grayscale paleodata as proxy of exit from Younger Dryas as used in Dakos et al (2008), see source below.

\end{Details}
%
\begin{Source}\relax
Dakos et al (2008), Slowing down as an early warning signal for abrup climate change, PNAS 105(38), 14308-14312.

\end{Source}
\printindex{}
\end{document}
